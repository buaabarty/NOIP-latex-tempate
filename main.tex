\documentclass[UTF8,a4paper]{ctexart}
\usepackage{geometry}
\usepackage{multicol}
\usepackage{multirow}
\usepackage{tabu}
\usepackage{xeCJK}
\usepackage{CJK}     
\usepackage{xeCJKfntef}                     
\usepackage{fancyhdr}               
\usepackage{graphicx}                 
\usepackage{lastpage}    
\usepackage{listings}
\usepackage{xcolor}
\usepackage{fontspec}
\usepackage{layout}
\usepackage{titletoc}
\usepackage[colorlinks,linkcolor=blue]{hyperref} 
\newcommand\filename[1]{\emph{\textbf{#1}}}
\newcommand\udot[1]{\textbf{\color{black}\CJKunderdot{\color{black}#1}}} % 第一个 color 调整加粗字体下着重号的颜色
\newcommand\newprob[2]{
    \newpage
    \pagestyle{fancy}
    \lhead{中文标题} \rhead{#1(#2)}
    \cfoot{第 \thepage 页 \qquad 共 \pageref{LastPage} 页}
    \phantomsection
    \addcontentsline{toc}{section}{#1(#2)}
    \begin{center}
        \LARGE
        \textbf{#1(}#2\textbf{)}
    \end{center}
    \large
    %
    \textbf{【题目描述】}
    \phantomsection
    \addcontentsline{toc}{subsection}{【题目描述】}
}
\newcommand\para[1]{
    $ $ \\ 
    \textbf{【#1】}
    \phantomsection
    \addcontentsline{toc}{subsection}{【#1】}
}
\newcommand\sample[2]{
    $ $ \\ 
    \textbf{【样例} #1\textbf{#2】}
    \phantomsection
    \addcontentsline{toc}{subsection}{【样例 #1 #2】}
}
\lstset{
    basicstyle={      
        \color{black}
        \fontspec{Courier New}
    },
    keywordstyle={
        \color{blue}
        \fontspec{Courier New}
    },
    numberstyle={
        \color{gray}
        \texttt
    },
    rulecolor=\color{blue},
    numbers=left,                               
    frame=single,                            
    % frameround=tttt,
    morekeywords={Sample, Input, Output},   % 可以手动添加关键字
}
\setmonofont{Courier New}
\geometry{left=2.52cm,right=2.52cm,top=2.5cm,bottom=2.5cm}
\begin{document}
    \pagestyle{fancy}
    \lhead{中文标题} \rhead{}
    \cfoot{第 \thepage 页 \qquad 共 \pageref{LastPage} \color{black} 页}
    \thispagestyle{empty}
    \addcontentsline{toc}{section}{注意事项}
    \begin{center}
        \Huge
        \textbf{中文标题}
        \\
        \Huge 
        EnglishContestName
        \\
        \huge
        \textit{SecondTitle}
        \\
        \Large
        \textbf{时间:}2021\textbf{年}1\textbf{月}1\textbf{日} 08:30 $\sim$ 1:00
        \\
    \end{center}
    \large
    \begin{center}
        \begin{tabular}{|p{3.1cm}|p{2.5cm}|p{2.5cm}|p{2.5cm}|p{2.5cm}|}
        \hline
        题目名称 & First & Second & Third & Forth \\
        \hline
        题目类型 & 传统型 & 传统型 & 传统型 & 传统型 \\
        \hline
        目录 & \texttt{first} & \texttt{second} & \texttt{third} & \texttt{fourth} \\
        \hline
        可执行文件名 & \texttt{first} & \texttt{second} & \texttt{third} & \texttt{fourth} \\
        \hline
        输入文件名 & \texttt{first.in} & \texttt{second.in} & \texttt{third.in} & \texttt{fourth.in} \\
        \hline
        输出文件名 & \texttt{first.out} & \texttt{second.out} & \texttt{third.out} & \texttt{fourth.out} \\
        \hline
        每个测试点时限 & 1.0 秒 & 1.0 秒 & 1.0 秒 & 1.0 秒 \\
        \hline
        内存限制 & 256 MiB & 256 MiB & 256 MiB & 256 MiB \\
        \hline
        测试点数目 & 5 & 5 & 5 & 5 \\
        \hline
        测试点是否等分 & 是 & 是 & 是 & 是 \\
        \hline
        \end{tabular}
    \end{center}
提交源程序文件名
    \begin{center}
        \begin{tabular}{|p{3.1cm}|p{2.5cm}|p{2.5cm}|p{2.5cm}|p{2.5cm}|}
        \hline
        对于 C++ \  语言 & \texttt{first.cpp} & \texttt{second.cpp} & \texttt{third.cpp} & \texttt{fourth.cpp} \\
        \hline
        \end{tabular}
    \end{center}
编译选项
    \begin{center}
        \begin{tabular}{|p{3.1cm}|p{11.2cm}<\centering|}
        \hline
        对于 C++ \ 语言 & \texttt{-O2 -std=c++14 -static} \\
        \hline
        \end{tabular}
    \end{center}
    \udot{注意事项(请仔细阅读)} 
    \\
    \indent
    1. 文件名(程序名和输入输出文件名)必须使用英文小写。\par
    2. C/C++ 中函数 main() 的返回值类型必须是 int,程序正常结束时的返回值必须是 0。\par
    3. 提交的程序代码文件的放置位置请参照具体要求。\par
    4. 因违反以上三点而出现的错误或问题,申诉时一律不予受理。\par
    5. 若无特殊说明,结果的比较方式为全文比较(过滤行末空格及文末回车)。\par
    6. 选手提交的程序源文件必须不大于 100KB。\par
    7. 程序可使用的栈内存空间限制与题目的内存限制一致。\par
    8. 只提供 Linux 格式附加样例文件。\par
    9. 评测在 NOI Linux 下进行,各语言的编译器版本以其为准。
    %%%%%%%%%%%%%%%%%%%%%%%%%%%%%%%%%%%%%%%%%%%%%%%%%%%%%%%%%%%%%%%%
\newprob{First}{first} \\ \indent
Description. \par
求出\udot{至少}有多少个数满足条件。

\para{输入格式} \\ \indent
从文件 \filename{first.in} 中读入数据。 \par
包含一个正整数 $n$。

\para{输出格式} \\ \indent
输出到文件 \filename{first.out} 中。 \par
Output. 

\sample{1}{输入}
\begin{lstlisting}
 Sample Input.
\end{lstlisting}

\sample{1}{输出}
\begin{lstlisting}
 Sample Output.
\end{lstlisting}

\sample{1}{解释} \\ \indent
Note.

\sample{2}{} \\ \indent
见选手目录下的 \filename{first/first2.in} 与 \filename{first/first2.ans}。

\para{数据范围} \\ \indent
\begin{center}
    \begin{tabu}{c|c|c|c}
        \tabucline[2pt]{-}
        测试点编号 & $n \le$ & $m \le$ & 特殊限制 \\ \tabucline[1.2pt]{-}
        $1 \sim 4$ & $15$ & $2000$ & \multirow{2}{*}{无} \\ \cline{1-3}
        $5 \sim 8$ & $2000$ & $15$ &  \\ \hline
        $9 \sim 12$ & \multicolumn{2}{c|}{$100$} & $c_i = C_j = 1$ \\ \hline
        $13 \sim 16$ & \multicolumn{2}{c|}{\multirow{3}{*}{$2000$}} & $w_i = W_j = 1$ \\ \cline{1-1} \cline{4-4}
        $17 \sim 20$ & \multicolumn{2}{c|}{} & $v_i = V_j = 1$ \\ \cline{1-1} \cline{4-4}
        $21 \sim 25$ & \multicolumn{2}{c|}{} & 无 \\ \tabucline[2pt]{-}
    \end{tabu}
\end{center}
    %%%%%%%%%%%%%%%%%%%%%%%%%%%%%%%%%%%%%%%%%%%%%%%%%%%%%%%%%%%%%%%%
\newprob{Second}{second} \\ \indent
Description. 

\para{输入格式} \\ \indent
从文件 \filename{second.in} 中读入数据。 \par

\para{输出格式} \\ \indent
输出到文件 \filename{second.out} 中。 \par
Output. 

\sample{1}{输入}
\begin{lstlisting}
 Sample Input.
\end{lstlisting}

\sample{1}{输出}
\begin{lstlisting}
 Sample Output.
\end{lstlisting}

\sample{1}{解释} \\ \indent
Note.

\sample{2}{} \\ \indent
见选手目录下的 \filename{second/second2.in} 与 \filename{second/second2.ans}。

\para{数据范围} \\ \indent
Constraint.
    %%%%%%%%%%%%%%%%%%%%%%%%%%%%%%%%%%%%%%%%%%%%%%%%%%%%%%%%%%%%%%%%%%%%%%%
\newprob{Third}{third} \\ \indent
Description. 

\para{输入格式} \\ \indent
从文件 \filename{third.in} 中读入数据。 \par

\para{输出格式} \\ \indent
输出到文件 \filename{third.out} 中。 \par
Output. 

\sample{1}{输入}
\begin{lstlisting}
 Sample Input.
\end{lstlisting}

\sample{1}{输出}
\begin{lstlisting}
 Sample Output.
\end{lstlisting}

\sample{1}{解释} \\ \indent
Note.

\sample{2}{} \\ \indent
见选手目录下的 \filename{third/third2.in} 与 \filename{third/third2.ans}。

\para{数据范围} \\ \indent
Constraint.
    %%%%%%%%%%%%%%%%%%%%%%%%%%%%%%%%%%%%%%%%%%%%%%%%%%%%%%%%%%%%%%%%%%%%%%%%%%%
\newprob{Fourth}{fourth} \\ \indent
Description. 

\para{输入格式} \\ \indent
从文件 \filename{fourth.in} 中读入数据。 \par

\para{输出格式} \\ \indent
输出到文件 \filename{fourth.out} 中。 \par
Output. 

\sample{1}{输入}
\begin{lstlisting}
 Sample Input.
\end{lstlisting}

\sample{1}{输出}
\begin{lstlisting}
 Sample Output.
\end{lstlisting}

\sample{1}{解释} \\ \indent
Note.

\sample{2}{} \\ \indent
见选手目录下的 \filename{fourth/fourth2.in} 与 \filename{fourth/fourth2.ans}。

\para{数据范围} \\ \indent
Constraint.
\end{document}